\documentclass[11pt, oneside]{article}   	% use "amsart" instead of "article" for AMSLaTeX format
\usepackage{geometry}                		% See geometry.pdf to learn the layout options. There are lots.
\geometry{a4paper}                   		% ... or a4paper or a5paper or ... 
%\geometry{landscape}                		% Activate for for rotated page geometry
%\usepackage[parfill]{parskip}    		% Activate to begin paragraphs with an empty line rather than an indent
\usepackage{graphicx}				% Use pdf, png, jpg, or eps with pdflatex; use eps in DVI mode
								% TeX will automatically convert eps --> pdf in pdflatex		
\usepackage{amssymb}

\title{ It's important to take notes }
\author{M.-H, Vitalii, Yasmine}
%\date{}							% Activate to display a given date or no date

\begin{document}
\maketitle
%\section{}
%\subsection{}

%*********************************************************
\section {General and random considerations}
What do we know about the  $\Lambda_b\to pK l^+l^-$ beasts  ?

\begin{itemize}
\item $\Lambda_b\to J/\psi (\to\mu^+ \mu^+) pK $ observed. Used to measure the lifetime of $\Lambda_b$ this mode is also used for pentaquark studies.
\item $\Lambda_b\to \psi(2S) (\to \mu^+ \mu^+) pK $ observed, check B\&Q paper.
\item $\Lambda_b\to \mu^+ \mu^- pK $ never observed (work ongoing P.Griffith and co). 
\item $\Lambda_b\to J/\psi/\psi(2S) (\to e ^+e^-) pK $ never observed. 
\item $\Lambda_b\to e ^+e^- pK $ never observed. 
\item $\Lambda_b\to \gamma(\to e ^+e^-) pK $ never observed. 
\item $q^2$ range goes from $2m_l^2$ to 16.96 GeV/$c^{2}$  %5619.51- 493.677 -100.7276466812 = 4118.55
\end{itemize}
What do we want to do : 

\begin{itemize}
\item Measure the lepton universality in $\Lambda_b\to pK l^+l^-$, let's call it $R_{\Lambda^*}$
\item Measure Branching ratio of $\Lambda_b \to p K \gamma $ with conversions. 

\end{itemize}




%**************
\paragraph {Back of the envelope Calculation of the expected yields :}

%**************
\paragraph {What do we know about the $pK$ spectrum?}


\paragraph {Questions ?}
\begin{itemize}
\item For the LeptonU measurement, how many $q^2$ bins can we afford  ? 
\item What range of $pK$ should we use ? 
\item If we want to do an angular analysis, how do we define the angles we care about ? Can we adapt whatever comes out from P2VV tuple tool ? What was used in the pentaquark paper ?
\end{itemize}


\begin{table}[h]
\centering
\begin{tabular}{l|c|c}
Channel & Yields & Reference \\
\hline
$B^0 \to J/\psi (\to\mu^+\mu^-) K^*$ & & \\
$B^0 \to J/\psi (\to e^+e^-) K^*$ & & \\
$B^0 \to K^*\mu^+\mu^- $ & & \\
$B^0 \to K^* e^+e^- $ & & \\
$B^0 \to \gamma(\to e^+e^-) K^*$ & & \\
\hline 
$\Lambda_b \to J/\psi (\to\mu^+\mu^-) pK$ & 26k/29k & arXiv:1507.03414v2/1603.06961v1\\
$\Lambda_b \to \psi(2S) (\to\mu^+\mu^-) pK$ &  665 & arXiv:1603.06961v1\\

$\Lambda_b \to J/\psi (\to e^+e^-) pK$ & & \\
$\Lambda_b pK \to \mu^+\mu^- $ & & \\
$\Lambda_b pK \to e^+e^- $ & & \\
$\Lambda_b\to \gamma(\to e^+e^-) pK$ & & \\
\end{tabular}
\caption { Measured and estimated yields.}
\end{table}


%*********************************************************
\section {Samples}
\label{sec:samples}
\begin{table}[h]
\centering
\begin{tabular}{llcr}
Sample& Event Type  & Information & Processed \\

\hline
$\Lambda_b\to \Lambda(1520) e^+e^-$ & 15124001 & Sim08 ? & 476 221 \\
$\Lambda_b\to pK e^+e^-$&15124011  &  Sim08 ? & 497 919 \\
$\Lambda_b\to J/\psi (e^+e^-)pK $&15154001  & Sim08 ?  & 1 214 792\\
$\Lambda_b\to \Lambda(1520) \gamma $ &15102201 &  Sim08 ? &383 997 \\
\hline 
$ B_s\to \phi \gamma$& 13102201 &  Sim08 ?  &   3 039 979\\

$B^0 \to K^* \gamma $ &11102201 &  Sim08 ? & 3 027 980\\

$ B^0 \to K^* (e^+e^-)$ &  11124001& Sim08 ?  &  1 272 496\\
\end{tabular}
\caption {Monte Carlo samples  - 2012 }
\end{table}

%*********************************************************
\section {Stripping selection}
\label{sec:stripping}
For the preliminary studies, data processed with  {\tt Stripping 21}, {\tt Reco 14} were used. 
In {\tt Stripping 21} the {\tt Bu2LLK}  stripping line selects the following  final states : $K, K*^*, \phi$. So we added in {\tt S21r0,1p1}\footnote{ incremental stripping of Run I data}$ \Lambda, \Lambda^*(\to pK ), K_S $ etc.



%*********************************************************
\section{Offline Selection}
\subsection{BDT Selection}
\label{sec:bdt}

\begin {table}[h]
\centering
\begin{tabular} {cc}
Variables & Importance \\
\hline

\end{tabular}
\caption{Variables used in the BDT.}
\end{table}


\subsection{PID Selection}
\label{sec:bdt}



\section{Papers and Useful references}
\begin{itemize}
\item The pentaquark paper : http://arxiv.org/abs/1507.03414

\end{itemize}



\end{document}  